\documentclass[11pt]{scrartcl}
\usepackage{evan}
\usepackage{booktabs}
\addtolength{\textheight}{1em}

\begin{document}
\title{Digression on type safety}
\subtitle{18.02 Recitation (MW9)}
\author{Evan Chen (\mailto{evan@evanchen.cc})}
\date{4 September 2024}
%\date{21 September 2022}
\maketitle

In mathematics, statements are usually either true or false.
Examples of false statements\footnote{Indiana Pi bill and 1984, respectively.} include
\[ \pi = \frac{16}{5} \qquad \text{or}\qquad 2+2=5. \]
However, it's possible to write statements that are not merely false,
but not even ``grammatically correct'', such as the nonsense equations
\[ \pi = \begin{pmatrix} 1 & 0 \\ 0 & 1 \end{pmatrix},
  \qquad \left\lvert \mathbf i + 3 \mathbf j \right\rvert = \cos(\mathbf k), \qquad
\det \begin{pmatrix} 5 \\ 11 \end{pmatrix} \neq \sqrt{2}.  \]
To call these equations false is misleading.
If your friend asked you whether $2+2=5$, you would say ``no''.
But if your friend asked whether $\pi$ equals the $2 \times 2$ identity matrix,
the answer is a different kind of ``no''; really, it's ``your question makes no sense''.

These three examples are \alert{type errors}.
This term comes from programming: most programming languages have different data types
like integer, boolean, string, array, etc.,
and will usually\footnote{JavaScript is a notable exception.
  In JavaScript, you may know that \texttt{[]} and \texttt{\{\}}
  are an empty array and an empty object, respectively.
  Then \texttt{[]+[]} is the empty string,
  \texttt{[]+\{\}} is the string \texttt{`[object Object]'},
  \texttt{\{\}+[]} is \texttt{0},
  and \texttt{\{\}+\{\}} is \texttt{NaN} (not a number).}
prevent you from doing anything idiotic like adding a string to an array.

Objects in mathematics work in a really similar way.
In the first weeks of 18.02, you will meet real numbers, vectors, and matrices;
these are all different types of objects, and certain operations are
\alert{defined} (aka ``allowed'') or \alert{undefined} (aka ``not allowed'')
depending on the underlying types.
Table~\ref{tab:operations} lists some common examples.

\begin{table}[h]
  \centering
  \begin{tabular}{lclll}
    \toprule
    Operation & Notation & Input 1 & Input 2 & Output \\ \midrule
    Add/subtract & $a+b$ & Scalar & Scalar & Scalar \\
    Add/subtract & $\vec v + \vec w$ & Length $d$ vector & Length $d$ vector & Length $d$ vector \\
    Add/subtract & $M+N$ & $m \times n$ matrix & $m \times n$ matrix & $m \times n$ matrix \\
    Multiplication & $c \vec v$ & Scalar & Length $d$ vector & Length $d$ vector \\
    Multiplication & $ab$ & Scalar & Scalar & Scalar \\
    Multiplication & $MN$ & $m \times n$ matrix & $n \times p$ matrix & $m \times p$ matrix \\
    Dot product & $\vec v \cdot \vec w$ & Length $d$ vector & Length $d$ vector & Scalar \\
    Cross product & $\vec v \times \vec w$ & Length $3$ vector & Length $3$ vector & Length $3$ vector \\
    Length/mag. & $\left\lvert \vec v \right\rvert$ & Any vector && Scalar \\
    Determinant & $\det A$ & Any square matrix && Scalar \\ \bottomrule
  \end{tabular}
  \caption{Common linalg operations.
    For 18.02, ``scalar'' and ``real number'' are synonyms.}
  \label{tab:operations}
\end{table}

A common question at this point is how you are supposed to figure out
whether a certain operation is allowed or not.
For example, many students want to try and multiply two vectors together component-wise;
why is
\[ \begin{pmatrix} 2 \\ 3 \end{pmatrix} \begin{pmatrix} 4 \\ 5 \end{pmatrix}
  \overset{?}{=} \begin{pmatrix} 8 \\ 15 \end{pmatrix} \]
not a legal sentence? It seems like it would make sense.

The answer is that you \emph{don't} have to figure out --- you are \emph{told}.
Table~\ref{tab:operations} isn't something that you derive.
Instead, it's the set of \emph{definitions} which you have been given.
Or more sarcastically, it's all just a social construct.\footnote{Of course,
  it's not quite as arbitrary as it sounds.
  The long explanation goes something like this.
  Definitions aren't judged by ``correctness'', because that doesn't even make sense;
  you are allowed to make up whatever definitions you want.
  Instead, definitions are judged by whether they are \emph{useful}.
  Which is obviously subjective, but it's less subjective than you might guess.}

\section*{Why you should care}
What this means is that, every time you encounter a new kind of
mathematical object or operation (e.g.\ partial derivative),
one of the first things that you might want to do is figure out
what kind of types it is permitted to interact with.
This helps give you a sanity check on your understanding of the new concept.

Practically, what's really useful is that if you have a good handle on types,
then it gives you way to do \alert{type checking} on your work.
This is the analog of dimensional analysis from 8.01/8.02, where
you know you messed up if some equation has
$\text{kg} \cdot \text{m} \cdot \text{s}^{-2}$ on the left but
$\text{kg} \cdot \text{m} \cdot \text{s}^{-1}$ on the right.

For example, if you are checking your work and you see something like
\[ \left\lvert \vec v \times \vec p \right\rvert = 9 \vec p \]
then you can immediately tell that there's a mistake,
because the two sides are incompatible --- the left-hand side is a real number,
but the right-hand side is a vector.

\section*{References}
\begin{itemize}
  \ii \emph{The Type System of Mathematics}, Qiaochu Yuan,
  \url{https://qchu.wordpress.com/2013/05/28/the-type-system-of-mathematics/}.
  \ii \emph{Basic Category Theory}, Tom Leinster, Chapter 3.
  \url{https://arxiv.org/abs/1612.09375}
\end{itemize}

\section*{Hyperlink}
You can find a copy of this file at
\url{https://web.evanchen.cc/1802.html}.
I'll also upload the slides for today there.


\end{document}
