\documentclass[11pt]{article}
\usepackage{amsmath,amsthm,amssymb}
\usepackage[colorlinks]{hyperref}
\usepackage{tikz, pgfplots}
\pgfplotsset{compat=1.17}

\begin{document}
\title{Quick answer key to Recitation 19}
\author{ChatGPT 4o}
\date{25 November 2024}
\maketitle

Use the table of contents below to skip to a specific part
without seeing spoilers to the other parts.

I just used ChatGPT to write this one quickly.
ChatGPT can make mistakes, so if you spot anything that's wrong, flag me to ask.

\tableofcontents



\newpage

\section{Solution}

We are tasked with expressing the triple integral
\[
\iiint_{R} f \, dV
\]
as iterated integrals in both spherical and cylindrical coordinates, where \( R \) is the region bounded below by the cone \( z = \sqrt{x^2 + y^2} \) and above by the upper hemisphere of radius \( 2 \), defined by \( x^2 + y^2 + z^2 \leq 4 \) with \( z \geq 0 \).

\newpage

\subsection{Understanding the Region \( R \)}

\subsubsection{Description of \( R \)}
- **Lower Bound:** The cone \( z = \sqrt{x^2 + y^2} \) can be rewritten as \( z^2 = x^2 + y^2 \), representing a double-napped cone. Since \( z \geq 0 \), we consider only the upper nappe of the cone.
- **Upper Bound:** The upper hemisphere \( x^2 + y^2 + z^2 = 4 \) with \( z \geq 0 \).
- **Intersection:** To find the limits of integration, determine where the cone and the hemisphere intersect.

\subsubsection{Finding the Intersection of the Cone and Hemisphere}

Set the equations equal to find the boundary of \( R \):
\[
z^2 = x^2 + y^2 \quad \text{and} \quad x^2 + y^2 + z^2 = 4
\]
Substitute \( z^2 = x^2 + y^2 \) into the hemisphere equation:
\[
x^2 + y^2 + (x^2 + y^2) = 4 \\
2(x^2 + y^2) = 4 \\
x^2 + y^2 = 2
\]
Thus, the cone and the hemisphere intersect along the circle \( x^2 + y^2 = 2 \) and \( z = \sqrt{2} \).

\newpage

\subsection{Part (a): Expressing the Integral in Spherical Coordinates}

\subsubsection{Spherical Coordinates Overview}
Spherical coordinates \((\rho, \theta, \phi)\) are related to Cartesian coordinates by:
\[
x = \rho \sin\phi \cos\theta, \quad y = \rho \sin\phi \sin\theta, \quad z = \rho \cos\phi
\]
where:
\begin{itemize}
    \item \( \rho \geq 0 \) is the radial distance,
    \item \( 0 \leq \theta < 2\pi \) is the azimuthal angle,
    \item \( 0 \leq \phi \leq \pi \) is the polar angle.
\end{itemize}
The volume element in spherical coordinates is:
\[
dV = \rho^2 \sin\phi \, d\rho \, d\phi \, d\theta
\]

\subsubsection{Determining the Limits of Integration}

- **Radial Distance (\( \rho \)):** The hemisphere has a radius of \( 2 \), so \( 0 \leq \rho \leq 2 \).
- **Azimuthal Angle (\( \theta \)):** Full rotation around the \( z \)-axis, so \( 0 \leq \theta < 2\pi \).
- **Polar Angle (\( \phi \)):** Determined by the cone \( z = \sqrt{x^2 + y^2} \).

  The cone equation in spherical coordinates:
  \[
  \rho \cos\phi = \rho \sin\phi \\
  \cos\phi = \sin\phi \\
  \tan\phi = 1 \\
  \phi = \frac{\pi}{4}
  \]
  Therefore, \( \phi \) ranges from \( 0 \) to \( \frac{\pi}{4} \).

\subsubsection{Setting Up the Integral}

The triple integral in spherical coordinates becomes:
\[
\iiint_{R} f \, dV = \int_{0}^{2\pi} \int_{0}^{\frac{\pi}{4}} \int_{0}^{2} f(\rho, \theta, \phi) \rho^2 \sin\phi \, d\rho \, d\phi \, d\theta
\]
where \( f(\rho, \theta, \phi) \) is the function expressed in spherical coordinates.

\newpage

\subsection{Part (b): Expressing the Integral in Cylindrical Coordinates}

\subsubsection{Cylindrical Coordinates Overview}
Cylindrical coordinates \((r, \theta, z)\) are related to Cartesian coordinates by:
\[
x = r \cos\theta, \quad y = r \sin\theta, \quad z = z
\]
where:
\begin{itemize}
    \item \( r \geq 0 \) is the radial distance,
    \item \( 0 \leq \theta < 2\pi \) is the azimuthal angle,
    \item \( z \) is the height.
\end{itemize}
The volume element in cylindrical coordinates is:
\[
dV = r \, dz \, dr \, d\theta
\]

\subsubsection{Determining the Limits of Integration}

- **Radial Distance (\( r \)):** The region is bounded by the cone and the hemisphere. At any height \( z \), the radial distance \( r \) satisfies:

  From the cone \( z = r \), and from the hemisphere \( r^2 + z^2 = 4 \), so \( r = \sqrt{4 - z^2} \).

- **Height (\( z \)):** The cone starts at \( z = 0 \) and goes up to the intersection point \( z = \sqrt{2} \).

- **Azimuthal Angle (\( \theta \)):** Full rotation around the \( z \)-axis, so \( 0 \leq \theta < 2\pi \).

Thus, the limits are:
\[
0 \leq \theta < 2\pi \\
0 \leq z \leq \sqrt{2} \\
z \leq r \leq \sqrt{4 - z^2}
\]

\subsubsection{Setting Up the Integral}

The triple integral in cylindrical coordinates becomes:
\[
\iiint_{R} f \, dV = \int_{0}^{2\pi} \int_{0}^{\sqrt{2}} \int_{z}^{\sqrt{4 - z^2}} f(r, \theta, z) \, r \, dr \, dz \, d\theta
\]
where \( f(r, \theta, z) \) is the function expressed in cylindrical coordinates.

\subsection{Conclusion}

The triple integral \( \iiint_{R} f \, dV \) over the region \( R \) can be expressed as iterated integrals in both spherical and cylindrical coordinates as follows:

\begin{enumerate}
    \item[(a)] \textbf{Spherical Coordinates}:
    \[
    \iiint_{R} f \, dV = \int_{0}^{2\pi} \int_{0}^{\frac{\pi}{4}} \int_{0}^{2} f(\rho, \theta, \phi) \rho^2 \sin\phi \, d\rho \, d\phi \, d\theta
    \]

    \item[(b)] \textbf{Cylindrical Coordinates}:
    \[
    \iiint_{R} f \, dV = \int_{0}^{2\pi} \int_{0}^{\sqrt{2}} \int_{z}^{\sqrt{4 - z^2}} f(r, \theta, z) \, r \, dr \, dz \, d\theta
    \]
\end{enumerate}

These iterated integrals allow for the evaluation of the triple integral over the specified region \( R \) using the most convenient coordinate system based on the symmetry of the region and the function \( f \).




\newpage

\section{Center of Mass of a Hemisphere}

We are tasked with finding the center of mass of a solid hemisphere of radius \( a \) using spherical coordinates. The hemisphere is assumed to have a uniform density \( \delta = 1 \).

\newpage

\subsection{Understanding the Problem}

Consider the upper hemisphere defined by:
\[
z \geq 0 \quad \text{and} \quad x^2 + y^2 + z^2 \leq a^2.
\]
Due to the symmetry of the hemisphere about the \( z \)-axis, the coordinates of the center of mass will satisfy:
\[
\overline{x} = 0, \quad \overline{y} = 0.
\]
Thus, we only need to determine the \( z \)-coordinate of the center of mass, \( \overline{z} \).

\newpage

\subsection{Formula for Center of Mass}

The \( z \)-coordinate of the center of mass for a solid region \( R \) with density \( \delta = 1 \) is given by:
\[
\overline{z} = \frac{1}{M} \iiint_{R} z \, dV,
\]
where \( M \) is the mass of the hemisphere:
\[
M = \iiint_{R} dV.
\]

\newpage

\subsection{Setting Up the Integral in Spherical Coordinates}

Spherical coordinates \((r, \theta, \phi)\) are related to Cartesian coordinates by:
\[
x = r \sin\phi \cos\theta, \quad y = r \sin\phi \sin\theta, \quad z = r \cos\phi,
\]
where:
\begin{itemize}
    \item \( r \geq 0 \) is the radial distance,
    \item \( 0 \leq \theta < 2\pi \) is the azimuthal angle,
    \item \( 0 \leq \phi \leq \pi \) is the polar angle.
\end{itemize}
The volume element in spherical coordinates is:
\[
dV = r^2 \sin\phi \, dr \, d\phi \, d\theta.
\]

\subsubsection{Limits of Integration}

For the upper hemisphere:
\[
0 \leq r \leq a, \quad 0 \leq \theta < 2\pi, \quad 0 \leq \phi \leq \frac{\pi}{2}.
\]

\newpage

\subsection{Calculating the Mass \( M \)}

First, compute the mass \( M \):
\[
M = \iiint_{R} dV = \int_{0}^{2\pi} \int_{0}^{\frac{\pi}{2}} \int_{0}^{a} r^2 \sin\phi \, dr \, d\phi \, d\theta.
\]

\subsubsection{Evaluating the Integral}

1. **Integrate with respect to \( r \):**
\[
\int_{0}^{a} r^2 \, dr = \left[ \frac{r^3}{3} \right]_{0}^{a} = \frac{a^3}{3}.
\]

2. **Integrate with respect to \( \phi \):**
\[
\int_{0}^{\frac{\pi}{2}} \sin\phi \, d\phi = \left[ -\cos\phi \right]_{0}^{\frac{\pi}{2}} = -\cos\left(\frac{\pi}{2}\right) + \cos(0) = 0 + 1 = 1.
\]

3. **Integrate with respect to \( \theta \):**
\[
\int_{0}^{2\pi} d\theta = 2\pi.
\]

4. **Combine the results:**
\[
M = \frac{a^3}{3} \times 1 \times 2\pi = \frac{2\pi a^3}{3}.
\]

\newpage

\subsection{Calculating triple integral}

Next, compute the integral \( \iiint_{R} z \, dV \):
\[
\iiint_{R} z \, dV = \int_{0}^{2\pi} \int_{0}^{\frac{\pi}{2}} \int_{0}^{a} (r \cos\phi) r^2 \sin\phi \, dr \, d\phi \, d\theta.
\]

\subsubsection{Simplifying the Integrand}

\[
z \, dV = (r \cos\phi) \cdot r^2 \sin\phi \, dr \, d\phi \, d\theta = r^3 \cos\phi \sin\phi \, dr \, d\phi \, d\theta.
\]

\subsubsection{Evaluating the Integral}

1. **Integrate with respect to \( r \):**
\[
\int_{0}^{a} r^3 \, dr = \left[ \frac{r^4}{4} \right]_{0}^{a} = \frac{a^4}{4}.
\]

2. **Integrate with respect to \( \phi \):**
\[
\int_{0}^{\frac{\pi}{2}} \cos\phi \sin\phi \, d\phi.
\]
Use the substitution \( u = \sin\phi \), \( du = \cos\phi \, d\phi \):
\[
\int_{0}^{\frac{\pi}{2}} u \, du = \left[ \frac{u^2}{2} \right]_{0}^{1} = \frac{1}{2}.
\]

3. **Integrate with respect to \( \theta \):**
\[
\int_{0}^{2\pi} d\theta = 2\pi.
\]

4. **Combine the results:**
\[
\iiint_{R} z \, dV = \frac{a^4}{4} \times \frac{1}{2} \times 2\pi = \frac{a^4 \pi}{4}.
\]

\newpage

\subsection{Calculating the Center of Mass \( \overline{z} \)}

Using the formula:
\[
\overline{z} = \frac{1}{M} \iiint_{R} z \, dV = \frac{\frac{a^4 \pi}{4}}{\frac{2\pi a^3}{3}} = \frac{a^4 \pi}{4} \times \frac{3}{2\pi a^3} = \frac{3a}{8}.
\]

\newpage

\subsection{Conclusion}

The center of mass of the hemisphere of radius \( a \) with uniform density is located at:
\[
\left( \overline{x}, \overline{y}, \overline{z} \right) = \left( 0, 0, \frac{3a}{8} \right).
\]




\newpage

\section{Gravitational Attraction of a Region \( R \) on a Unit Mass at the Origin}

We are tasked with finding the gravitational attraction of the region \( R \) bounded above by the plane \( z = 2 \) and below by the cone \( z^2 = 4(x^2 + y^2) \), on a unit mass located at the origin. The region \( R \) has a constant density \( \delta = 1 \).

\newpage

\subsection{Understanding the Region \( R \)}

\subsubsection*{Description of \( R \)}
- **Lower Bound:** The cone \( z^2 = 4(x^2 + y^2) \) can be rewritten as \( z = 2\sqrt{x^2 + y^2} \) (considering \( z \geq 0 \)).
- **Upper Bound:** The plane \( z = 2 \).
- **Intersection:** To find the boundary of \( R \), set \( z = 2\sqrt{x^2 + y^2} \) equal to \( z = 2 \):
  \[
  2\sqrt{x^2 + y^2} = 2 \implies \sqrt{x^2 + y^2} = 1 \implies x^2 + y^2 = 1
  \]
  Thus, the cone and the plane intersect along the circle \( x^2 + y^2 = 1 \) at \( z = 2 \).

\newpage

\subsection{Setting Up the Gravitational Attraction}

The gravitational attraction \( \mathbf{F} \) at the origin due to the mass distribution in \( R \) is given by:
\[
\mathbf{F} = -G \iiint_{R} \frac{\mathbf{r}}{|\mathbf{r}|^3} \delta \, dV
\]
where:
- \( G \) is the gravitational constant (assuming \( G = 1 \) for simplicity),
- \( \mathbf{r} = \langle x, y, z \rangle \) is the position vector of a point in \( R \),
- \( |\mathbf{r}| = \sqrt{x^2 + y^2 + z^2} \).

Due to the symmetry of the region \( R \) about the \( z \)-axis, the \( x \) and \( y \)-components of \( \mathbf{F} \) will cancel out, leaving only the \( z \)-component. Therefore, we focus on calculating the \( z \)-component of \( \mathbf{F} \), denoted as \( F_z \):
\[
F_z = -G \iiint_{R} \frac{z}{(x^2 + y^2 + z^2)^{3/2}} \, dV
\]
Assuming \( G = 1 \), we have:
\[
F_z = - \iiint_{R} \frac{z}{(x^2 + y^2 + z^2)^{3/2}} \, dV
\]

\newpage

\subsection{Converting to Cylindrical Coordinates}

To evaluate the integral, we convert to cylindrical coordinates \((r, \theta, z)\), where:
\[
x = r\cos\theta, \quad y = r\sin\theta, \quad z = z
\]
The volume element in cylindrical coordinates is:
\[
dV = r \, dz \, dr \, d\theta
\]
The integrand becomes:
\[
\frac{z}{(r^2 + z^2)^{3/2}}
\]

\subsubsection*{Determining the Limits of Integration}
- **Radial Distance \( r \):** For a fixed \( z \), \( r \) ranges from \( 0 \) up to where the cone and plane intersect:
  \[
  z = 2r \implies r = \frac{z}{2}
  \]
  However, since the plane \( z = 2 \) bounds \( z \), \( r \) ranges from \( 0 \) to \( \frac{z}{2} \).
  
- **Height \( z \):** Ranges from the base of the cone \( z = 0 \) up to the plane \( z = 2 \).
  
- **Azimuthal Angle \( \theta \):** Full rotation around the \( z \)-axis, \( 0 \leq \theta < 2\pi \).

Thus, the limits are:
\[
0 \leq \theta < 2\pi, \quad 0 \leq z \leq 2, \quad 0 \leq r \leq \frac{z}{2}
\]

\newpage

\subsection{Setting Up the Integral}

The \( z \)-component of the gravitational attraction is:
\[
F_z = - \int_{0}^{2\pi} \int_{0}^{2} \int_{0}^{\frac{z}{2}} \frac{z}{(r^2 + z^2)^{3/2}} \cdot r \, dr \, dz \, d\theta
\]

\newpage

\subsection{Evaluating the Integral}

\subsubsection*{Step 1: Integrate with Respect to \( r \)}
Consider the inner integral:
\[
I_r = \int_{0}^{\frac{z}{2}} \frac{z \cdot r}{(r^2 + z^2)^{3/2}} \, dr
\]
Let \( u = r^2 + z^2 \), then \( du = 2r \, dr \), so \( r \, dr = \frac{du}{2} \).

Substituting:
\[
I_r = z \cdot \frac{1}{2} \int_{u = z^2}^{u = z^2 + \left(\frac{z}{2}\right)^2} u^{-3/2} \, du = \frac{z}{2} \left[ -2u^{-1/2} \right]_{z^2}^{\frac{5z^2}{4}} = \frac{z}{2} \left( -2 \cdot \frac{1}{\sqrt{\frac{5z^2}{4}}} + 2 \cdot \frac{1}{\sqrt{z^2}} \right )
\]
Simplify:
\[
I_r = \frac{z}{2} \left( -2 \cdot \frac{2}{z\sqrt{5}} + 2 \cdot \frac{1}{z} \right ) = \frac{z}{2} \left( -\frac{4}{z\sqrt{5}} + \frac{2}{z} \right ) = \frac{z}{2} \cdot \frac{-4 + 2\sqrt{5}}{z\sqrt{5}} = \frac{-4 + 2\sqrt{5}}{2\sqrt{5}} = \frac{-2 + \sqrt{5}}{\sqrt{5}} = \frac{\sqrt{5} - 2}{\sqrt{5}}
\]

\subsubsection*{Step 2: Integrate with Respect to \( z \)}
Now, the integral becomes:
\[
F_z = - \int_{0}^{2\pi} \int_{0}^{2} \frac{\sqrt{5} - 2}{\sqrt{5}} \, dz \, d\theta = - \frac{\sqrt{5} - 2}{\sqrt{5}} \int_{0}^{2\pi} \int_{0}^{2} dz \, d\theta
\]
Evaluate the integrals:
\[
\int_{0}^{2} dz = 2, \quad \int_{0}^{2\pi} d\theta = 2\pi
\]
Thus:
\[
F_z = - \frac{\sqrt{5} - 2}{\sqrt{5}} \cdot 2 \cdot 2\pi = -4\pi \cdot \frac{\sqrt{5} - 2}{\sqrt{5}} = -4\pi \left( \frac{\sqrt{5} - 2}{\sqrt{5}} \right )
\]

\newpage

\subsection{Simplifying the Expression}

Rationalize the denominator:
\[
F_z = -4\pi \left( \frac{\sqrt{5} - 2}{\sqrt{5}} \right ) = -4\pi \left( \frac{(\sqrt{5} - 2)}{\sqrt{5}} \cdot \frac{\sqrt{5}}{\sqrt{5}} \right ) = -4\pi \left( \frac{5 - 2\sqrt{5}}{5} \right ) = -4\pi \left( 1 - \frac{2\sqrt{5}}{5} \right ) = -4\pi + \frac{8\pi}{\sqrt{5}}
\]
Alternatively, leaving it in the original form is also acceptable:
\[
F_z = -4\pi \left( 1 - \frac{2}{\sqrt{5}} \right )
\]

\newpage

\subsection{Final Answer}

The gravitational attraction of the region \( R \) on a unit mass at the origin is directed along the negative \( z \)-axis and has a magnitude of:
\[
\mathbf{F} = \left( 0, 0, -4\pi \left( 1 - \frac{2}{\sqrt{5}} \right ) \right )
\]
Alternatively, rationalized:
\[
\mathbf{F} = \left( 0, 0, -4\pi \left( \frac{\sqrt{5} - 2}{\sqrt{5}} \right ) \right )
\]
This vector represents the gravitational force exerted by the region \( R \) on the unit mass at the origin.


\end{document}
