\documentclass[11pt]{article}
\usepackage{amsmath,amsthm,amssymb}
\usepackage[colorlinks]{hyperref}

\begin{document}
\title{Quick answer key to Recitation 9}
\author{ChatGPT 4o}
\date{1 October 2024}
\maketitle

Use the table of contents below to skip to a specific part
without seeing spoilers to the other parts.

I just used ChatGPT to write this one quickly.
ChatGPT can make mistakes, so if you spot anything that's wrong, flag me to ask.

\tableofcontents



\newpage

\section{Solution}

The acceleration vector of the ball is given by:
\[
\mathbf{a}(t) = 2 \mathbf{i} - 32 \mathbf{k}
\]
The initial velocity vector of the ball is:
\[
\mathbf{v}(0) = 80 \mathbf{j} + 80 \mathbf{k}
\]

We are tasked with finding the velocity vector \( \mathbf{v}(t) \), the position vector \( \mathbf{r}(t) \), and the speed of the ball when it hits the ground.

\subsection{Part 1: Find \( \mathbf{v}(t) \) (the velocity vector)}

To find the velocity vector \( \mathbf{v}(t) \), we integrate the acceleration vector \( \mathbf{a}(t) \) with respect to time \( t \):
\[
\mathbf{v}(t) = \int \mathbf{a}(t) \, dt = \int (2 \mathbf{i} - 32 \mathbf{k}) \, dt
\]
\[
\mathbf{v}(t) = 2t \mathbf{i} - 32t \mathbf{k} + \mathbf{C}
\]
where \( \mathbf{C} \) is the constant vector of integration.

Using the initial condition \( \mathbf{v}(0) = 80 \mathbf{j} + 80 \mathbf{k} \), we can solve for \( \mathbf{C} \). At \( t = 0 \):
\[
\mathbf{v}(0) = \mathbf{C} = 0 \mathbf{i} + 80 \mathbf{j} + 80 \mathbf{k}
\]
Thus, the velocity vector is:
\[
\mathbf{v}(t) = 2t \mathbf{i} + 80 \mathbf{j} + (80 - 32t) \mathbf{k}
\]

\newpage

\subsection{Part 2: Find \( \mathbf{r}(t) \) (the position vector)}

To find the position vector \( \mathbf{r}(t) \), we integrate the velocity vector \( \mathbf{v}(t) \):
\[
\mathbf{r}(t) = \int \mathbf{v}(t) \, dt = \int (2t \mathbf{i} + 80 \mathbf{j} + (80 - 32t) \mathbf{k}) \, dt
\]
\[
\mathbf{r}(t) = \left( t^2 \mathbf{i} \right) + \left( 80t \mathbf{j} \right) + \left( 80t - 16t^2 \right) \mathbf{k} + \mathbf{D}
\]
where \( \mathbf{D} \) is the constant vector of integration.

Using the initial condition that the ball starts at the origin, \( \mathbf{r}(0) = \mathbf{0} \), we solve for \( \mathbf{D} \). At \( t = 0 \):
\[
\mathbf{r}(0) = \mathbf{D} = \mathbf{0}
\]
Thus, the position vector is:
\[
\mathbf{r}(t) = t^2 \mathbf{i} + 80t \mathbf{j} + (80t - 16t^2) \mathbf{k}
\]

\newpage

\subsection{Part 3: Find the speed when the ball hits the ground}

The ball hits the ground when its \( z \)-coordinate (the \( \mathbf{k} \)-component) is zero. From the position vector \( \mathbf{r}(t) \), the \( z \)-coordinate is:
\[
z(t) = 80t - 16t^2
\]
Setting \( z(t) = 0 \) to find the time when the ball hits the ground:
\[
80t - 16t^2 = 0
\]
\[
t(80 - 16t) = 0
\]
\[
t = 0 \quad \text{or} \quad t = \frac{80}{16} = 5
\]

The ball hits the ground at \( t = 5 \) seconds. To find the speed at that moment, we calculate the magnitude of the velocity vector \( \mathbf{v}(t) \) at \( t = 5 \).

The velocity vector at \( t = 5 \) is:
\[
\mathbf{v}(5) = 2(5) \mathbf{i} + 80 \mathbf{j} + (80 - 32(5)) \mathbf{k} = 10 \mathbf{i} + 80 \mathbf{j} + (80 - 160) \mathbf{k} = 10 \mathbf{i} + 80 \mathbf{j} - 80 \mathbf{k}
\]

The speed is the magnitude of the velocity vector:
\[
\text{Speed} = |\mathbf{v}(5)| = \sqrt{10^2 + 80^2 + (-80)^2} = \sqrt{100 + 6400 + 6400} = \sqrt{12900}
\]
\[
\text{Speed} = \sqrt{12900} \approx 113.6 \, \text{feet per second}
\]

\newpage

\subsection{Conclusion}

1. The velocity vector is:
\[
\mathbf{v}(t) = 2t \mathbf{i} + 80 \mathbf{j} + (80 - 32t) \mathbf{k}
\]
2. The position vector is:
\[
\mathbf{r}(t) = t^2 \mathbf{i} + 80t \mathbf{j} + (80t - 16t^2) \mathbf{k}
\]
3. The speed of the ball when it hits the ground is approximately \( 113.6 \, \text{feet per second} \).




\newpage

\section{Solution}

\newpage

\subsection{Trajectory 1: \( \mathbf{r}(t) = \langle t^2, t^3 \rangle \) for \( 0 \leq t \leq 2 \)}

To find the total distance traveled by the particle along the trajectory, we compute the arc length of the curve given by \( \mathbf{r}(t) \). The formula for arc length is:
\[
L = \int_{a}^{b} |\mathbf{v}(t)| \, dt
\]
where \( \mathbf{v}(t) = \frac{d\mathbf{r}(t)}{dt} \) is the velocity vector and \( |\mathbf{v}(t)| \) is its magnitude.

First, calculate the velocity vector \( \mathbf{v}(t) \):
\[
\mathbf{v}(t) = \frac{d}{dt} \langle t^2, t^3 \rangle = \langle 2t, 3t^2 \rangle
\]

Now, find the magnitude of the velocity vector:
\[
|\mathbf{v}(t)| = \sqrt{(2t)^2 + (3t^2)^2} = \sqrt{4t^2 + 9t^4}
\]

The total distance traveled by the particle is the arc length:
\[
L = \int_{0}^{2} \sqrt{4t^2 + 9t^4} \, dt
\]
Factor the expression inside the square root:
\[
L = \int_{0}^{2} t \sqrt{4 + 9t^2} \, dt
\]

To solve this, we use substitution. Let \( u = 4 + 9t^2 \), so \( du = 18t \, dt \). Then:
\[
L = \frac{1}{18} \int_{4}^{40} \sqrt{u} \, du = \frac{1}{18} \left[ \frac{2}{3} u^{3/2} \right]_{4}^{40}
\]
\[
L = \frac{1}{18} \times \frac{2}{3} \left[ 40^{3/2} - 4^{3/2} \right]
= \frac{1}{27} \left[ 40^{3/2} - 8 \right]
\]

Thus, the total distance traveled by the particle along the trajectory is approximately:
\[
L \approx \frac{1}{27} \times (253.26 - 8) \approx 9.08 \text{ units}
\]

\newpage

\subsection{Trajectory 2: \( \mathbf{r}(t) = \langle 2 \cos(3t), 2 \sin(3t) \rangle \) for \( 0 \leq t \leq 2\pi \)}

Again, we compute the total distance traveled by the particle using the arc length formula.

First, calculate the velocity vector \( \mathbf{v}(t) \):
\[
\mathbf{v}(t) = \frac{d}{dt} \langle 2 \cos(3t), 2 \sin(3t) \rangle = \langle -6 \sin(3t), 6 \cos(3t) \rangle
\]

Now, find the magnitude of the velocity vector:
\[
|\mathbf{v}(t)| = \sqrt{(-6 \sin(3t))^2 + (6 \cos(3t))^2} = \sqrt{36 \sin^2(3t) + 36 \cos^2(3t)} = \sqrt{36 (\sin^2(3t) + \cos^2(3t))} = 6
\]

The magnitude of the velocity vector is constant, so the total distance traveled is:
\[
L = \int_{0}^{2\pi} 6 \, dt = 6 \times 2\pi = 12\pi \, \text{units}
\]

\newpage

\subsection{Comparison with the Length of the Curve on the \( xy \)-Plane}

The given trajectory \( \mathbf{r}(t) = \langle 2 \cos(3t), 2 \sin(3t) \rangle \) represents a parameterization of a circle with radius 2. The particle traces this circle three times as \( t \) runs from \( 0 \) to \( 2\pi \).

The circumference of a circle with radius 2 is:
\[
\text{Circumference} = 2\pi \times 2 = 4\pi \, \text{units}
\]

Since the particle completes three full circles, the total length of the curve is:
\[
\text{Length} = 3 \times 4\pi = 12\pi \, \text{units}
\]

This matches the total distance traveled by the particle, confirming that the particle moves along a circular path with radius 2 and completes three full revolutions.




\newpage

\section{Solution}

\subsection{Part 1: Show that a particle moves at constant speed if and only if its velocity vector and acceleration vector are always perpendicular}

Let \( \mathbf{r}(t) \) be the position vector of a particle at time \( t \), and let \( \mathbf{v}(t) = \frac{d\mathbf{r}(t)}{dt} \) be its velocity vector and \( \mathbf{a}(t) = \frac{d\mathbf{v}(t)}{dt} \) be its acceleration vector.

\paragraph{Forward Direction:}
We assume that the particle moves at constant speed and show that the velocity vector and acceleration vector are always perpendicular.

If the particle moves at constant speed, then the magnitude of the velocity vector \( |\mathbf{v}(t)| \) is constant. Let \( v(t) = |\mathbf{v}(t)| \). Since the speed is constant, we have:
\[
\frac{d}{dt} \left( v(t) \right) = \frac{d}{dt} \left( |\mathbf{v}(t)| \right) = 0
\]

Now, recall that the speed is the magnitude of the velocity vector:
\[
v(t)^2 = \mathbf{v}(t) \cdot \mathbf{v}(t)
\]
Differentiating both sides with respect to time, we obtain:
\[
\frac{d}{dt} \left( v(t)^2 \right) = \frac{d}{dt} \left( \mathbf{v}(t) \cdot \mathbf{v}(t) \right)
\]
\[
0 = 2 \mathbf{v}(t) \cdot \mathbf{a}(t)
\]

This implies:
\[
\mathbf{v}(t) \cdot \mathbf{a}(t) = 0
\]

Thus, the velocity vector and acceleration vector are always perpendicular when the particle moves at constant speed.

\paragraph{Reverse Direction:}
Now, assume that the velocity vector and acceleration vector are always perpendicular, i.e., \( \mathbf{v}(t) \cdot \mathbf{a}(t) = 0 \), and show that the particle moves at constant speed.

Since \( \mathbf{v}(t) \cdot \mathbf{a}(t) = 0 \), we have:
\[
\frac{d}{dt} \left( \mathbf{v}(t) \cdot \mathbf{v}(t) \right) = 2 \mathbf{v}(t) \cdot \mathbf{a}(t) = 0
\]

This implies that \( \mathbf{v}(t) \cdot \mathbf{v}(t) \), which is the square of the speed \( v(t)^2 \), is constant. Therefore, the speed \( v(t) \) is constant.

Thus, a particle moves at constant speed if and only if its velocity vector and acceleration vector are always perpendicular.

\newpage

\subsection{Part 2: Suppose that the position vector and acceleration vector are always proportional to each other; show that \( \mathbf{r}(t) \times \mathbf{v}(t) \) is a constant vector}

Let the position vector \( \mathbf{r}(t) \) and acceleration vector \( \mathbf{a}(t) \) be proportional to each other. This means there exists a scalar function \( \lambda(t) \) such that:
\[
\mathbf{a}(t) = \lambda(t) \mathbf{r}(t)
\]

We want to show that the vector \( \mathbf{r}(t) \times \mathbf{v}(t) \) is constant.

First, recall that the acceleration vector \( \mathbf{a}(t) \) is the derivative of the velocity vector:
\[
\mathbf{a}(t) = \frac{d\mathbf{v}(t)}{dt}
\]

To prove that \( \mathbf{r}(t) \times \mathbf{v}(t) \) is constant, we differentiate \( \mathbf{r}(t) \times \mathbf{v}(t) \) with respect to time:
\[
\frac{d}{dt} \left( \mathbf{r}(t) \times \mathbf{v}(t) \right) = \frac{d\mathbf{r}(t)}{dt} \times \mathbf{v}(t) + \mathbf{r}(t) \times \frac{d\mathbf{v}(t)}{dt}
\]
\[
= \mathbf{v}(t) \times \mathbf{v}(t) + \mathbf{r}(t) \times \mathbf{a}(t)
\]

Since the cross product of any vector with itself is zero, the first term vanishes:
\[
\mathbf{v}(t) \times \mathbf{v}(t) = 0
\]

Using the assumption that \( \mathbf{a}(t) = \lambda(t) \mathbf{r}(t) \), the second term becomes:
\[
\mathbf{r}(t) \times \mathbf{a}(t) = \mathbf{r}(t) \times \left( \lambda(t) \mathbf{r}(t) \right) = \lambda(t) \left( \mathbf{r}(t) \times \mathbf{r}(t) \right) = 0
\]

Thus:
\[
\frac{d}{dt} \left( \mathbf{r}(t) \times \mathbf{v}(t) \right) = 0
\]

This shows that \( \mathbf{r}(t) \times \mathbf{v}(t) \) is a constant vector.


\end{document}
